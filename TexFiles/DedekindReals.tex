\documentclass[ProjectReport]{subfiles}

\begin{document}

\section{Dedekind Reals}
\subsection{Goals}

In my third year project, I defined the Dedekind reals, and showed that there is an embedding from rationals to reals. The definition is given. 

%TC:ignore
\RealsCode
%TC:endignore

Note that in the definition of the reals, we use of the truncated sigma type \AgdaFunction{$\exists$} and truncated disjunction type \AgdaFunction{$\vee$}. This is necessary, because we cannot decide if arbitrary real numbers are equal. 

The natural goal is to show that this definition of the Reals, equipped with the usual addition and multiplication operators satisfies the constructive field axioms, and further show that the field is complete and Archimedean. This describes the reals as we view them intuitively, as a continuous line with no gaps stretching infinitely in two directions.

\begin{itemize}
    \item Define addition, multiplication and inverse operations
    \item Define order, apartness relations
    \item Show that the Dedekind reals equipped with the above satisfy the constructive field axioms
    \item Show that the Dedekind reals are Archimedean
    \item Show that the Dedekind reals are Dedekind-complete
\end{itemize}

\subsection{Order and Apartness Relations}

The definitions for order and non-strict order are intuitive. I define them as in the HoTT Book \cite[Section 11.2.1]{hottbook}. 

\RealsOrder

The notion behind strict order is that if we can find some rational between two reals $x$ and $y$, then $x < y$. Note the use of the overloaded operators; for strict order we have two distinct relations which compare rationals and reals. The first says that $q$ is in the upper cut of $x$, and the second that says $q$ is in the lower cut of $y$. This is useful syntactic sugar, since it is more intuitive to think of a rational being less or greater than a real number, rather than belonging to the lower or upper cut of a real. 

%TODO: AGDA-FY IT!

The following proof is an example of what a proof may look like when we have terms whose types are Dedekind Reals.

%TODO: R<-trans PROOF

Unsurprisingly, it is not much different to a standard proof. Recalling the definition of $x < y$ and $y < z$, we obtain rationals $k$ and $l$, except they are propositionally truncated. Since we want to return some value that is also truncated, we can simply provide a function which ignores the truncation to \AgdaFunction{∥∥-functor}. We can also escape a truncation if the return type of a function is a proposition. A reader with a background in functional programming may recognise that propositional truncation is a monad, and proofs in Agda with truncated types proceed similarly to how one would write monadic functions. Do notation is available in Agda, however in this development I avoid this as I find it to be less readable, and if I struggle to read my own code I cannot expect a reader to be able to decipher it.

The Archimedean property is has many equivalent definition, but I choose to define it in the same way as the HoTT book \cite[Theorem 11.2.6]{hottbook}. 

%TODO : Add Archimedean Proof.

My reasoning for choosing this over any other definition (you can view a few variations here \cite{noauthor_archimedean_2022}), is that the proof is automatic, simply by the definition of strict order. An extension of this project could define more variations of the property and verify they are equivalent in the presence of the same field axioms.

We now move onto the apartness relation.

%TODO : Add APARTNESS Relation

When defining complete Archimedean ordered field in classical mathematics, one doesn't need an apartness relation. The reason we need this relation is due to one of the classical field axioms not holding constructively. Specifically, in a field we have that for every real $x$ not equal to $0$, $x$ has a multiplicative inverse $x'$ such that $x * x' = 1$. This is not sufficient constructively, Troelstra and Dalen say \cite[Page 256]{troelstra1988constructivism} that ``apartness is a positive version of inequality...". This is analogous to how a classically one might say that a set is ``non-empty", whereas a constructively one would say that a set is ``inhabited".

By defining apartness as above, we retain enough information so that we can prove the field conditions, whilst preserving the ``x not equal to zero" property.

I also proved some properties of order of the Reals:

%TODO 


\subsection{Operations}

At this stage, we have to decide how to implement operations of reals. Specifically, at the very least we need addition and multiplication. We would also like negation, for the additive inverse of a Real. We hope to avoid division, since I avoided this when formalising my rationals, although in the long term I would like to add division as an operator for both Reals and Rationals.

%TODO : ADD REFERENCE

As I discussed in my last project, I would like to avoid directly defining the operations; for each one we have to produce a Dedekind cut which satisfies the four conditions of a Dedekind Real, which is shown to be cumbersome by the simple proof that the rationals are embedded in the reals. 

Avoiding direct implementation of the operators is further justified by my first attempt at defining addition directly. Addition on the Dedekind Reals is defined as a cut with the following conditions (as in the HoTT Book \cite{hottbook}:

%TODO add dedekind reals conditions

Inhabitedness, roundedness and disjointedness are shown to be satisfied fairly easily; each follows from the respective property on $x$ and $y$, for example inhabitedness of $z$ follows from inhabitedness of $x$ and $y$.

%TODO ADD CODE HERE

Locatedness is not so easily provable. We want to show that for any $a < b$, either there exists $p < x, u < y$ such that $a = p + u$, or there exists $x < q, y < v$ such that $b = q + v$. As a naive first approach, since the other conditions follow from their respective conditions on $x$ and $y$, we look at what locatedness of $x$ and $y$ gives us. We are given $a < b$, so we obtain that $a < x$ or $x < b$, and $a < y$ or $y < b$. One might stare at these facts for a long time before realising that this is not enough information to prove the locatedness of $z$. 

Bauer and Taylor remark in ``The Dedekind reals in abstract Stone duality" \cite[Remark 11.14]{bauer_taylor_2009} that ``Locatedness always seems to be the most difficult, because we need to calculate the result arbitrarily closely". 

The problem is that we don't have enough information about how \textit{close} $a$ and $b$ are to either $x$ or $y$. We can move arbitrarily small distances to the left $a$, since $x$ is rounded and for all $\epsilon > 0$, $a - \epsilon < a$, but we cannot guarantee that $\epsilon < y$. By locatedness of $y$, we can find that $a - \epsilon < y$ or $y < a$. In the first case we need that either $\epsilon < x$, and in the second we have no extra useful information. 

To tackle this problem, I used an idea from the same paper \cite[Proposition 11.15]{bauer_taylor_2009}. This proposition introduces the idea of Arithmetic Locatedness of the reals.

%TODO ADD CODE FOR ARITHMETIC LOCATEDNESS%.

The intuition behind Arithmetic Locatedness of the reals is that we can locate a real number $x$ to any degree of precision we wish. For any $\epsilon > 0$, we can an interval $(p , q)$ with $x$ located somewhere in this interval. If we had Arithmetic Locatedness of the reals, then we could apply \cite[Lemma 11.16]{bauer_taylor_2009} to complete the addition operator. By writing Arithmetic Locatedness in Agda, leaving the proof as a hole, I managed to finish the implementation of addition directly.

%TODO: Add code for addition, cleaned up

The next step is clearly to prove that the reals are arithmetically located. A full discussion can be found later in this report, but for now it suffices to say that the first attempt was not successful. With the experience of implementing the embedding from \AgdaFunction{ℚ} to \AgdaFunction{ℝ}, and addition of reals sans the proof of arithmetic locatedness, I decided it was worth trying a different approach to implement the operations.

\subsection{Extensions}

Extensions of functions are an intuitive concept. Suppose we have sets $X, Y$ and $Z$, with $X \subset Y$, and a function $f : X \to Z$. A function $g : Y \to Z$ is considered an extension of $f$ if $g$ agrees with $f$ for every $x \in X$. We would like to implement addition, multiplication and inverses on the reals, and we notice that we already have these operations for reals, and that the rationals may be embedded in the reals. The question is, then, can we extend these functions from the rationals to the reals?

%This paragraph is likely not needed

One study of extensions is ``McShane-Whitney extensions in constructive analysis" \cite{petrakis2020mcshane} which discusses more complicated extensions than I am interested in, and is a dense read for my current knowledge of constructive mathematics, but there is an topical note in the introduction that says ''general extension theorems are rare and usually have interesting topological consequences". 

Usually by imposing some conditions on the function we want to extend, we can find a unique extension to the new space. In this section I will discuss two extensions.My supervisor provided me with a reference to Simmons, "Introduction to Topology and Modern Analysis", with the goal of working towards Theorem D, \cite[Page 78]{simmons1983introduction}. The theorem is as follows:

Let $X$ be a metric space, let $Y$ be a complete metric space, and let $A$ be a dense subspace of $X$. If f is a uniformly continuous mapping of $A$ into $Y$, then $f$ can be extended uniquely to a uniformly continuous mapping $g$ of $X$ into $Y$.

In our specific case, $X$ and $Y$ are both the Reals, and $A$ is the rationals. This is known as the continuous extension theorem. The idea is that we can take continuous extensions from the rationals to the reals, and uniquely extend them to functions from the reals to the reals. This is exactly what we need to show that the Dedekind Reals form a complete ordered field; by take our addition, multiplication and inverse functions defined for the rationals and extending them, I avoid having to directly implement the operations. With some luck, since I have already proven that the rationals are field, by extending the functions in this way the field axioms also follow more easily than proving them for the operations implemented directly.

On the other hand, the proof above mentions concepts which I haven't yet formalised. In order to prove the continuous extension theorem, I need to consider metric spaces, continuity, cauchy and convergent sequences, completeness of a metric space. These are all familiar concepts, but a daunting prospect to formalise in Agda. 

As a proof of the concept of the extension of functions, I will now discuss a different extension, which I will call the monotonic extension theorem. 

\subsection{Monotonic Extension Theorem}

This theorem was suggested to me by my supervisor, who sketched out the proof for me in a Google Jamboard. If $f$ is rational valued function on the rationals, and $f$ is a monotonic function, and there exists a function $g$ where $f$ and $g$ are bijective, then there exists an unique extension $\hat{f}$ where $\hat{f}$ is a real valued function on the reals.

We begin by proving a lemma. If $f$ and $g$ are bijective, and $f$ is strictly monotone, then $g$ is strictly monotone.

%TODO: Insert agda proof here.

With this, we can write a function which produces an extension $\hat{f}$ given monotonic function $f$ with a bijection. 

%TODO : Add this proof

And finally, we can prove that this function preserves the behaviour of $f$. That is, $\hat{f} \circ \iota \equiv \iota \circ f$, where $\iota : \AgdaFunction{Q} \to \AgdaFunction{R}$ is the canonical embedding of \AgdaFunction{Q} in \AgdaFunction{Q}.

%TODO : ADD THIS PROOF

This proof of concept was successful, I proved that given a monotonic function with a bijection on the rationals, I can extend it to the reals. Of course, there is a reason that this is a proof of concept and not the end of the story; montonicity is a very strong condition. There are many functions which are not monotone, with multiplication being a particular example. This theorem is not strong enough for me to obtain the operations I need to prove that the Dedekind Reals are a field, and so we look back to the continuous extension theorem. Multiplication is continuous, so with that we should be able to extend multiplication of rationals to the reals.


As discussed before, in order to prove the continuous extension theorem, it's first necessary to formalise every concept included in the proof. We first turn our attention to metric spaces.

\section{Metric Spaces}

We immediately run into a problem when trying to define a metric space. A metric space is simply a set (which may be empty), with a distance function, known as a metric. There are many metrics one could choose from, however, usually a metric $d$ would be defined on a set $X$ with $d : X \times X \to \mathbb{R}$. One of the axioms of a metric space is the triangle inequality, which has that $d(x , z) \leq d(x,y) + d(y,z)$ for $x , y , z \in X$. But $d(x,y) + d(y,z)$ is addition of the reals. The point of implementing the continuous extension theorem was to \textit{avoid} directly implementing addition of the reals, so we need to consider a different approach.

Instead we replace the distance function onto reals with an equivalent definition which instead bounds %TODO DISCUSS METRIC SPACE

We also have to replace the axioms with equivalent axioms for the metric function. With this new definition, we can use metric space which are defined in terms of only rational numbers. 

%TODO AGDA CODE

%TODO FILL OUT THIS SECTION

Recalling the continuous extension theorem in Simmons, we require that both the rationals and the reals are metric spaces. Naturally, we begin with the rationals. 

\subsection{Rationals Metric Space}

The distance function for Rationals is taken to be the standard metric. Given $\epsilon > 0$, and $x , y \in \AgdaFunction{Q}$ we want to bound the standard distance between $x$ and $y$ smaller than $\epsilon$. 

To define this, it was necessary to extend my implementation of the rationals to include absolute values, and formalise many properties of the absolute value function. 

%TODO, add code for absolute value functions.

With this, I can define the metric for the set of rationals, and prove that the rationals are a metric space with respect to the alternative axioms. It would be nice in the future to prove that the alternative definition of the metric space is equivalent to the standard definition. This is not necessary, and I think this stems from a desire to be similar to classical mathematics. 

I had imagined that absolute values would be difficult to work with, but I found that the proofs were intuitive and this section of the project progressed nicely. One proof I found amusing is the proof that the absolute value functions is distributive over multiplication. There are a lot of proofs that you can \textit{see}, without having to refer to literature, but I could not figure out a way to prove this, and after consulting text books realising it requires a simple case analysis of the inputs.

\subsection{Dedekind Reals Metric Space}

The first thing we need is a distance function for the Dedekind Reals. My supervisor knew of a suitable distance function. We say that two reals are $\epsilon$-close if we can find a pair of rationals, one either side of each real such that the the distance between the furthest value on each side is less than $\epsilon$. 

%TODO: ADD CODE AND PICTURE HERE

This definition is equivalent to the standard definition of the metric. The intuition behind the definition is that we locate the two reals to some level of precision, then the reals are at most as far as way as the bounds of the precision. For example, if $a < x < b$, and $c < y < d$, then $x$ and $y$ are within the interval $[min a c , max b d]$, so the difference between them is at most $| min a c - max b d |$.

To define this metric, I first need the min and max operations on rational numbers. It was surprisingly difficult to formalise these operations. This was a case where working within type theory complicated the implementation of the operations. I wanted to define the operations using $<$ and $\leq$, and the fact that rationals number are trichotomous with respect to order. 

%Add min max

%Defining min and max in this way means that I need to prove properties of min and max by first 

%https://github.com/adsneap/Reals/commit/96c706a0f3c60a81f2fb8e957aca9cf46abee08b#diff-ebfc21b3aa729aeacbde135384ac5e98f1878d8c588a1b74996f9940a01694e8R23

%Not sure whether to talk about this more.

It is left to prove that the reals are a metric space with respect to the above metric. The second and third conditions are easy to prove using order transitivity and commutativity of min and max. The other properties provided to be a particular challenge. It turns out that to prove m1b for the reals, I needed to prove that the reals are arithmetically located, which I abandoned earlier in my work due to it's difficulty.

\subsubsection{Arithmetic Locatedness of Reals}

Recall the definition of Arithmetic locatedness. Given an arbitrary real number $x$, we need to find a rational either side of it, with the rationals bounded by some arbitrarily small value.

Since $x$ is inhabited, we begin with two rationals either side of $x$ which can be arbitrarily far apart. The question is then, how do we get arbitrarily close to $x$, given rationals $p$ and $q$ with $p < x < q$? If $p$ or $q$ were guaranteed to be close to $x$, this wouldn't be an issue since we could simply choose $p$ and $p + \epsilon$.

A naive approach might take one side, and move closer repeatedly using roundedness of $x$, but if $q$ is greater than a distance $\epsilon$ away from $x$ this is not fruitful. As such, we need to use an iterative process to get close to $x$ on each side. One such process is described by Bauer and Taylor \cite[Proposition 11.15]{bauer_taylor_2009} uses approximate halves, along with a different (but equivalent) statement of the Archimedean principle. To avoid proving that the variations are equivalent, I instead chose to attempt my the proof using trisections of the interval $(p,q)$. This proof was suggested to me by Tom de Jong, a PHD student at the University of Birmingham, and the idea of the proof follows.

By trisecting the interval $(p,q)$, we obtain rationals $a,b$ with $p<a<b<q$, with each sub interval a third of the difference of the whole interval. By using locatedness of the interval, we see that $a < x$ or $b < x$. We now have a new interval, either $(p,b)$ or $(a,q)$, which is $2/3$ the size of the original interval. By repeatedly applying trisections, we will find an interval smaller than any arbitrary $\epsilon$. 

%INCORRECT. NEED 2/3^n < \epsilon
The crux of the proof relies on the following: given $a,\epsilon : \AgdaFunction{Q}$, with $a , \epsilon > 0$ there exists $n : N$ such that $\frac{2}{3}^n * a < \epsilon$. This statement is clearly true, but fiendishly difficult to prove. My supervisor suggested formalising logarithms, but after some initial work I decided that it would be too difficult for me to implement this way, and I temporarily abandoned the proof while I worked on Metric Spaces. After realising that I need this proof for the work on metric spaces too, I turned to it my full attention. I asked one of my Maths Lecturers if it was possible to solve this without using logarithms, and he came up with a nice solution using the binomial expansion of $2/3$. This was a nice proof (as an exercise, try and prove the fact using binomial expansions!), but I still believed this proof would be too difficult to implement in Agda, since I would need to formalise summations and manipulations of sums. After a lot of thought, I came up with a proof of my own, by using limits of sequences of Rational Numbers.

\subsubsection{Limits of Rational Sequences}

Given the Metric Space defined above, the limit is defined as it is in classical mathematics. 

%TODO Limit definition

By proving the sandwich theorem by for sequences of rational numbers, I proved that result that I needed. There were several intermediate results needed. Trivially, the constant sequence which maps natural numbers to the rational number $0$ converges to $0$. 

%TODO: AGDA CODE

With some algebraic work, I proved that $0$ is the limit of the sequence defined by $S : N \to Q$ with $S(n) = \frac{1}{n+1}$. We cannot choose $1/n$, because unfortunately the first term of the sequence the sandwich only occurs when the sequence $\frac{1}{n}$ is shifted by one.

%TODO: Add AGDA CODE

Now, I prove that this sequence is indeed sandwiched by the constant $0$ sequence and $\frac{1}{n+1}$. A problem arose during the proof of this function. I haven't identified the exact cause, but for anything more than very simple inequalities of rational numbers, trying to compile the code results in an infinite hang. With thanks to de Jong, the solution is to put any offending inequalities into an abstract sub-proof. This prevents Agda from unfolding the proof, which speeds up type-checking. 

This is enough to make the RationalsLimits file compile, but is not enough for some inequalities, and a further measure is required. By adding the a command line option to the Agda file: --experimental-lossy-unification \cite{lossy_unification}. Unfortunately, as the documentation describes, this is sound, but not complete. There are some expressions that no longer type-check that normally would with this option in place, however for my purposes it has allowed to compile files that otherwise don't. It is important to note that it doesn't allow anything to compile that shouldn't, so for now I view the necessity of the use of this option a limitation of Agda, and with more development on the language it may be be necessary to use the option in the future.

With the proof that this sequence converges, I can now finish the proof that the Dedekind Reals are Arithmetically Located. I can locate any real to any degree of error, bounding it below and above by rational approximations of the real. This is a major result, I suspect one of the biggest results related to the Dedekind Reals, and I put it to immediate use.

\subsubsection{Dedekind Reals Satisfy Axioms}

A large portion of the project was spent on proving that the Dedekind Reals are a complete metric space. Due to the nature of proofs which involve the Dedekind Reals, the proofs are long, and certainly difficult to understand without a description of proof alongside the Agda code. Unfortunately, due to time constraints I have been unable to provide this for most of the proofs in this project, and I will reflect on this in the discussion section of this report.

Conditions 2 and 3 are proved easily using properties of order and the metric for rational numbers. 

%Agda Code for Metric Space

Part b of property 1 follows easily from Arithmetic Locatedness of Reals.

%Agda Code

Part a of property 1 is more difficult, and requires a lot of algebraic manipulation. It also required me to learn how to prove that two reals are equal. Since reals are defined in terms of subsets of rational numbers, we need to determine when subsets of types are equal. Unfortunately, in this mathematical settings it is not possible to directly prove that two subsets are equal (unless they are definitionally equal). Similarly to function extensionality, we must postulate propositional extensionality.

%AGDA CODE

Given two propositions of a type, we can prove they are equal if we can establish an ``if and only if" relationship between inhabitants of the propositions. The Dedekind Reals are defined in terms of two subsets of rationals; the lower cut and the upper cut. Discussion within my supervisor resulted in a proof that allows me to establish equality of a Real, given that the lower subsets of each are equal. This is useful machinery to avoid duplicate code, since usually the proofs for the upper and lower cuts are symmetric.

%AGDA CODE

Now, it is possible to prove part a of property 1. 

%AGDA CODE

Property 4 was not an easy proof. I spent a long time trying to come up with an \textit{efficient} proof, since otherwise we have an explosion of cases, due to the use of min and max in the metric for Dedekind Reals. I did manage to the proof, but it is still very large. I suspect that it is possible to reduce the proof to a few lines, but in the interest of moving forward with the project I resorted to considering a lot of cases and directly proving each case. This results in a proof which is extremely difficult to read and follow, but as usual, since the proof type-checks it is certainly \textit{correct}.

\subsubsection{Completeness of Metric Space}

A complete metric space is a metric space in which every Cauchy sequence converges. The definition is encapsulated by the following types. 

%Agda Code

For the classical version of the CET, it is required that the target domain is a complete metric space. Assuming that I would need to have this assumption to prove the CET from rationals to Dedekind reals, I proved that the Dedekind reals are a complete metric space with respect to our chosen metric. 

I use the approach of the HoTT book, which first defines a Cauchy approximation sequence, and provides a partial proof that every Cauchy approximation sequence has a limit \cite[Theorem 11.2.12]{hottbook}.

%First, we define the limit. The idea is intuitive, 

The proof states that ``it is clear that $L_y$ and $U_y$ are inhabited, rounded and disjoint". While it was clear the author, it wasn't so clear to me, and I couldn't find any reference to a complete version of this proof. I devised a proof for my own for roundedness, but owe thanks to Ambridge for help with the proof of inhabitedness, and de Jong for the proof of disjointness.

For disjointness, it was helpful to prove a lemma, proving the equivalence of the two versions of disjointness in the definition of a Dedekind real. 

%TODO: Agda code

With the proof that this limit exists, it remains to show that it is indeed the limit of the Cauchy approximation sequence. This proof is given \cite{hottbook}, although due to my alternate definition of a metric space, I struggled to reformulate the proof for my own work. By slowly working through the proof on paper, I eventually managed to adapt it for my library.

By proving that every Cauchy sequence has a modulus of convergence, and that every Cauchy sequence satisfies some Cauchy approximation sequence (with help from my supervisor), I completed the proof that the Reals are a complete metric space.

- Continuous extension theorem (and possibilities)
- Direct implementation of addition, group with respect to addition
- Direct multiplication

\end{document}

