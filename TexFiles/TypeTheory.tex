\documentclass[ProjectReport]{subfiles}

\begin{document}

\section{Type Theory}
The mathematical framework within which to work is an important choice. Zermelo-Fraenkel Set Theory (ZFC) is dominant in the modern world, characterised by the axiom of choice. In summary, the axiom says that given a potentially infinite collection of inhabited sets, we can construct a new set by choosing one element from each set in the collection \cite{choiceaxiom}. The axiom gives rise to proofs that are otherwise not provable in the system. 

Constructive mathematics is a framework in which every proof provides a witness. This is clearly advantageous in terms of computation, since proofs written constructively provide an algorithm to obtain an inhabitant of the proof. With a classical proof utilising the axiom of choice, it is not so simple, and producing an algorithm may be very difficult or even impossible. Mathematicians who prefer constructive mathematics may argue that the axiom should not be used, since it is not constructive, and what use is a proof if you cannot provide an example of its application? 

In some constructive frameworks, by adding the axiom of choice to we obtain classical mathematics. One might view constructive mathematics as embedded in classical mathematics. Adopting constructive mathematics also has drawbacks. The proofs which are only true using the axiom of choice are lost. Usually, there are also constructive arguments which give the same proof, but not always. Constructive arguments may turn out to be more complex. 

Classical mathematics has had much more development, since constructive mathematics is a young field. There has been a lot of research in the area, since Brouwer begun pioneering his form of constructivism known as intuitionism in 1907 \cite{Brouwer}. For example, Bishop made major contributions to constructive analysis \cite{Bishop1987-BISCA-2} followed by Troelstra \cite{Troelstra1973-TROMIO}, while progress is being made by multiple authors in constructive algebra \cite{2015}.

Type theory is another young field living in the intersection of Mathematics, Computer science and logic. To avoid the growing number of paradoxes in mathematics, in the early 20th century Russel introduced a system known as Type Theory \cite{coquand_type_2018}. In the time since, there has been a huge amount of research in the field, and correspondences have been established between Type Theory, Mathematics and Logic. In 1980, Howard explicitly stated this correspondence \cite{Howard1980-HOWTFN-2}, which has since been extended to include category theory interpretations of Type Theory, and an even younger field known as Homotopy Type Theory (see \cite{hottbook}). 

One alternative foundation of mathematics is known as Intuitionistic Type Theory, or Martin-L\"{o}f Type Theory (MLTT) after its founder Per Martin L\"{o}f. This type theory is aiming to be to constructive mathematics what ZFC is to classical mathematics \cite{dybjer_intuitionistic_2020}. One model of MLTT is known as Univalent Type Theory, introduced by Voevodsky, and characterised by the Univalence axiom. 

The goal of Univalent Type Theory is pure; as Grayson put it, Univalent foundations may provide "a language for mathematics invariant under equivalence and thus freed from irrelevant details and able to merge the results of mathematicians taking different but equivalent approaches" \cite{grayson_introduction_2018}. The applications of such a foundations are clear; using proof assistants such as Agda and Coq to verify the \textit{correctness} of a proof eliminates this burden for mathematicians. Moreover, while the theory is born out of constructive thinking, it is not limited to constructive mathematics. By adding certain axioms to the system, one can obtain classical mathematics, allowing one to verify the correctness of classical proofs. 

The power of this system goes even further when considering the consequence of writing propositions as types. By constructing an inhabitant of this type, we have a proof of this type. But writing this proof in a proof assistant also gives us a program to compute it, which is unquestionably useful to mathematicians: by writing proofs in such a language we can produce algorithms which are verifiably correct. Widely accepted incorrect proofs \cite{Incorrect_Proofs} are not unheard of.

In my view, there is no correct answer in choosing which mathematical foundation to work in. I can understand why a classical mathematician would be reluctant to give up the axiom of choice or the law of excluded middle, but also why a constructive mathematician champions the benefits of constructive proofs. The motivation behind a Univalent foundation of mathematics is exciting to me, I can envision a world where all mathematical proofs are readily available, regardless of foundational system, in verifiably correct libraries, and with algorithms available for constructive proofs. While it may be more difficult to write certain proofs constructively, I think it is a small price to pay, and a classical mathematician should have no worries in using a constructive version of a proof, and may continue to work classically by adding the axiom of choice to Univalent Foundations.

There are developments of the Reals that exist, namely {TODO : ADD SOURCE} a Coq library by Andreij Bauer and a non-constructive Agda development.

With the above in mind, writing a constructive library for the Dedekind Reals is an exciting project for me. It poses various challenges, in that I cannot simply copy and paste classical proofs, and with a background of mostly classical mathematics it forces me to think in a constructive way. I have to take care with definitions to ensure they hold constructively, and adapt proofs where necessary to avoid classical arguments. I have the added benefit in that my supervisor Martin Escard\'o has an active development of a library Univalent Type Theory \cite{TypeTopology}, and so my work will contribute to the active development in a relatively new field. The code can be compiled to Haskell code, so my code may be used to write algorithms which need the Reals to an arbitrary degree of precision, which is not possible with the standard floating point representation of real numbers.

With the above as perspective, I will be writing this report somebody with a background in classical mathematics, and as such will remark throughout the report on proofs and definitions which are altered due to my choice to work constructively, perhaps with the hope to entice a reader to think in a more constructive way, or to do further reading in the field of constructive mathematics.

\end{document}