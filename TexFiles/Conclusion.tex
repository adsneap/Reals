\documentclass[ProjectReport]{subfiles}

\begin{document}

\section{Conclusion}

The initial goal of this project was to formalise the complete ordered Archimedean field in Agda. After direct attempts at defining operators for reals, it became clear that formalisation involving reals is laborious.  As such, I searched to find an approach which avoided direct implementation.

The continuous extension theorem was identified as a candidate alternative. Extension theorems have been widely studied, but there seems to be a lack of constructive proofs in literature. 

Hence, I formalised metrics spaces, using an alternative definition which avoids the use of reals, and proved that the Dedekind reals and rationals are metric spaces. This required formalising more operators of rationals and their properties, including the min, max and absolute value operations. It also required the particularly challenging proof that the reals are arithmetically located, which I worked on over the course of the project, and finally proved using the sandwich theorem for sequences of rationals numbers.

I proved that the reals are a complete metric space, which was the second most challenging proof in this project. In particular, proving that the limit of a Cauchy sequence satisfies disjointedness took weeks of thought, and assistance from Escardo, Ambridge and de Jong, even though it is claimed that is ``is clear" in literature.

I have developed a deeper appreciation of the importance of clean code. While it may be true that the proofs in this project \textit{correct}, it doesn't serve any pedagogical purpose if it is difficult to understand the idea behind the proof. I believe the cleanliness of code should be on the same pedestal as correctness. If the code is not self-documenting, then a description in mathematical prose should be provided. At various points in the project I became frustrated because I couldn't understand a proof in literature (or the proof wasn't given), and there is no excuse to pass on this frustration to readers of my own code. 

I have appreciated experimenting with my work flow. I have tried to work on different proofs when I am finding a particular one too difficult. Unfortunately, I fell into this trap towards the end of the project, working on the continuous extension theorem for too long, and making only small incremental progress. I realise that I need to make concerted efforts to acknowledge more precisely the best time to work on a different problem, to be more productive with my time.

I discussed the concept of minimal hypothesis, and how I let go the notion that my work should be ``mathematically pure" in favour of elegance and readability of proofs.

I have discussed the potential avenues for extensions of this project. Exact computation is an exciting extension, following the work of Booij. Formalisation of the real number section in the HoTT book is an interest of mine, as well as general applications of real numbers in real and complex analysis, and topology.

I would like to thank Escardo, Ambridge and de Jong, for helping to me learn and understand the ideas behind many of the proofs in this project, and helping to sketch pen and paper proofs for me to implement in Agda.

\end{document}