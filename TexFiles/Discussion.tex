\documentclass[ProjectReport]{subfiles}

\begin{document}

\section{Discussion}

\subsection{Progress}
Over the course of this project, I have formalised a large amount of work on this constructive library for Dedekind reals. I have proven that the Dedekind reals are a complete metric space, and the rationals are a metric space. I have shown that monotonic rational functions can be extended real functions. I have developed a framework to be able to prove that continuous functions can be extended. I have defined and prove that addition of reals is a group. I have defined order of reals and proved properties of order of real numbers. I have defined the limit of sequences of rationals, and proved the sandwich theorem for such sequences. I have proved that the reals are arithmetically located. In order to prove that the Reals are a complete metric space, I defined Cauchy and convergent sequences, and proved that every Cauchy sequence of real numbers converges to a limit. 

Due to the complexity of the proofs towards the end of this project, I did not achieve the ambition of proving that the reals form a complete ordered Archimedean field. Regardless, I am proud of the progress I have made thus far, and will continue the work outside of the project to achieve this goal. Two proofs that I am particularly proud of are the proof that the Reals are arithmetically located, and the prove that Cauchy approximation sequences converge. Both of these proofs required a large amount of research, pen and paper attempts and resilience. 

\subsection{Code Quality}

I have made efforts throughout the project to produce high quality code. It is important for code to be understandable, particularly if the code is a mathematical proof. I haven't been fully successful in this regard. There are two contributing factors. 

The first is that I was building on top of my work in my previous project, in which I was learning how to implement mathematical proofs in Agda, resulting in both inefficient proofs and general untidiness. 

The second is that the proofs towards the end of the project were generally very large, and it is not always easy to find the most elegant solution in mathematics. It is natural to quickly prove lemmas in Agda, forgetting to clean them up after finishing the solution. 

I have learned that I should be extremely diligent in this regard. I should ensure that after the completion of any proof, I check that the proof is in an acceptable state, and provide a mathematical description of the proof and the idea behind it.

\subsection{Prioritisation}

I have had a tendency to try and work on difficult problems for too long. I spent a few weeks early in the project trying to prove the Arithmetic Locatedness of the Reals. This was not a fruitful endeavor, I eventually abandoned the attempt. It would have been more productive to focus my work elsewhere sooner. 

I think there is a critical point where you should switch to working on a different problem. Each time I began working on a different problem throughout this project, I would make progress. I had to reconsider this proof later in the project, and with more experience working with the reals I was able to find a solution. 

This is a difficult balance. Working on proofs tends to be a sequential process, so working on something else usually means assuming that what you are currently working on is correct. It is entirely feasible to postulate proofs (by leaving it as a hole in Agda), and it turns out that it is impossible to prove. As a result, I have been reluctant to jump ahead of difficult proofs. 

I have learned that I should more swiftly identify when I should stop working on a difficult problem, and work productively on a different problem.

\subsection{Identifying Proofs}

I have found that constructive analysis has been particularly challenging. Certainly there has been a wealth of work on constructive analysis (see \cite{troelstra1988constructivism} or \cite{Bishop1987-BISCA-2}), but this work tends to be from a purely mathematical perspective, without references to Type Theory. As such, most of the proofs in my work have been constructed ``in my own words", sometimes using \textit{ideas} from proofs found in literature. 

An egregious example I mentioned earlier was in a proof in the HoTT book, which stated that ``it is clear that $L_y$ and $U_y$ are inhabited, rounded and disjoint". Unfortunately, it was not so clear to me, and it took a few weeks for me to work through the proof, with help from Ambridge, and de Jong who suggested a strategy for proving the disjointedness of the limit of Cauchy sequences. It can be frustrating to not understand a proof, whilst knowing that it can be solved. This exemplifies why it is extremely important for proofs to be clean, because otherwise they are pedagogically useless. 

On the other hand, it is extremely satisfying to write a proof which you have constructed on your own. I am very proud of my proof that there exists some $n$ such that $(\frac{2}{3})^n < p$ for any positive $p$. As I mentioned previously, I had multiple suggestions, but eventually came up with my own strategy using the sandwich theorem. It is satisfying regardless to construct a proof based on ideas found in literature, but even more so when you develop the proof completely on your own, and prove that your idea for the proof is correct. 

\subsection{Minimal Hypotheses}

Mathematically speaking, a theorem is stronger when it requires fewer assumptions, because it is then applicable to more situations. As such, a ``perfect" development would prioritise assuming the minimal hypothesis for each and every proof. 

Consider now a proof which I updated over the course of this new project. Originally, I formulated and \href{https://adsneap.github.io/CSProject/Rationals.html#27337}{proved it without the use of function extensionality}. 

%Insert the updated code here.

It is appreciable not only how the readability increases, but how it is much easier to understand how the proof works. The drawback is that in order to write the proof in the above way, I am required to assume function extensionality. In the first variation, there is a large amount of algebraic manipulation of integers, without a clear reason why this is necessary. In the updated proof it is clear exactly what the chosen point is, and how it is proved that it is between the two arbitrary rationals. 

Previously, I believed that the development should be ``pure", and that every proof should only assume the absolute minimum necessary to complete the proof, but after discovering situations such as the above, I find myself leaning towards the side of elegance. 

For the purposes of this project, it is not so important that my specific implementation of certain proofs involving rationals assumes function extensionality. 
Firstly, the rationals were built with the intention of being used for the Dedekind reals, which requires liberal use of function extensionality, and is globally assumed in files which use the Reals. Secondly, is it unlikely that these rationals will be used outside the context of the Dedekind reals. If I needed another version which explicitly excluded function extensionality, I could trivially formalise them.

\end{document}